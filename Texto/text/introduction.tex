\chapter{Introdução}
\label{chp:introduction}

% \begin{quotation}[]{Poul Anderson}
% I have yet to see any problem, however complicated, which, when looked at in the
% right way, did not become still more complicated.
% \end{quotation}

%acp acronimo no plural

%ver como colocar referencia só de ano na citação
%verificar se é indicado citação na introdução. O que a banca acha?
%verificar a quantidade mínima/máxima de parágrafos indicados
A cada ano novas aplicações e dispositivos com formatos e recursos diversos são lançados no mercado, o crescimento de aplicações como as de transporte, dos serviços de streamings, a popularização de  dispostivos IoT, o crescimento dos datacenters e do serviço em nuvem vêm contribuindo fortemente para o aumento na produção de dados \citep{relatorio-idc}.

Especializada em computação em nuvem, a \citeauthor{domo-study} realiza anualmente uma análise da quantidade de informação produzida na internet 
a cada \textit{minuto}, abaixo vemos algumas das estatísticas relativas à esse estudo \citet{domo-study}:
\begin{itemize}
    \item 41.666.667 mensagens são enviadas através do aplicativo de mensagens WhatsApp;
    \item 479.452 pessoas interagem com conteúdos da plataforma Reddit;
    \item 208.333 pessoas participam de conferências no serviço de chamada Zoom, ao passo que o Microsoft Teams conecta cerca de 52.083 usuários por minuto;
    \item 147.000 uploads de fotos são feitos no Facebook e 150.000 mensagens são enviadas na mesma plataforma;
    \item 28 faixas de música são incluídas no serviço de streaming Spotify a cada minuto;
    \item 500 horas de vídeos são enviadas ao youtube.
\end{itemize}

Prevê-se que essa quantidade de dados irá crescer nos próximos anos. De acordo com o relatório anual da \citet{relatorio-cisco}, em 2018 o número de
usuários conectados à internet era de 3,9 bilhões, e até 2023 mais de 70\% da população terá acesso à internet, atingido a marca de 5,3 bilhões de usuários.
A International Data Corporation (IDC) prevê que em 2025, 75\% (6 bilhões) da população mundial interaja com essas aplicações todos os dias,
fazendo com que a quantidade de dados produzidos cresça de 33 Zetabytes (ZBs) em 2018, para 175 ZBs em 2025 \citep{relatorio-idc}.% TODO trocar para citação numérica


Esee crescimento da produção de dados tem relação direta com o surgimento de novos dispositivos. O grande problema é que existe uma lacuna de
desempenho entre CPU e memória. Essa lacuna se deve principalmente ao fato de que os fabricantes de processadores estão focados em obter uma lógica rápida, 
que acelera a comunicação, ao passo que os fabricantes de memória objetivam uma capacidade de armazenamento de dados maior \citep{paper-processor-memory-bottleneck}.
Essa disparidade é um grande desafio para projetitas e desenvolvedores, pois como afirmam \citet{paper-processor-memory-bottleneck}, 
esta faz com que seja criado um atraso na comunicação entre CPU e memória (gargalo de Von-Neumann), sobretudo quando se trabalha com um grande conjunto
de dados. Diversas soluções foram e estão sendo desenvolvidas para contornar o problema dessa lacuna, entre elas o uso de computação paralela em nível
de instrução, hierarquia de memória multinível, \textit{smarter memories}, técnicas específicas de tolerância à latência, e compressão de dados.
\citep{paper-Processor-Memory-bottleneck-Problems-Solutions, paper-processor-memory-bottleneck}


%  Sendo amplamente utiliziadas em sistemas como os de mecanismo de busca ou sistemas que trabalham com dados geográficos, elas também são essenciais para lidar  com dispositivos onde a quantidade de memória é limitada, como  por exemplo dispositivos IoTs e outros embarcados.\\
Como mostra \cite{book-compact-data-structures}, estruturas de dados sucintas são uma forma de compressão de dados, estas representam a informação de
maneira eficiente, usando um espaço próximo ao limite inferior estabelecido pela Teoria da Informação, possibilitando ainda
operações sobre seus objetos de modo que não seja necessário a descompactação dos mesmos, o que a torna mais eficiente do que os algorotimos de compatação 
clássicos que precisam de armazenamento e tempo extra para descompactar os objetos sobre os quais operam. 
Esses fatores fazem com que as estruturas de dados sucintas sejam amplamente usadas em sistemas como os de mecanismo de busca, 
ou sistemas que trabalham com dados geográficos. Estruturas de dados sucintas também são essenciais para lidar  com dispositivos onde a quantidade de memória é limitada, como  dispositivos IoTs e outros embarcados \citep{book-compact-data-structures}.
Por fim, mesmo que seja necessário reter parte do conjunto de dados em um nível mais baixo da hierarquia de memória, onde o tempo para obter uma informação 
é maior, o uso de estruturas de dados sucintas ainda possibilitará aplicações eficientes, haja vista que o seu uso minimizará o número de acessos a 
memórias mais lentas.

Trabalharemos especificamente com o estudo de árvores sucintas, uma das estruturas mais populares no campo da Ciência da Computação, tanto no caso das representações clássicas quanto para as sucintas. De acordo com \citet{paper-succint-trees-in-practice}, as implementações de árvores sucintas existentes na literatura podem diferir em sua funcionalidade, variando daquelas que suportam navegação de um nó filho para um nó pai ou aquelas que suportam operações como obter o menor ancestral
comum de dois nós, até àquelas que suportam um conjunto completo de operações, podendo variar em relação ao espaço ocupado, indo  de $O(n/(\log \log n)^2)$ à $O(n/ polylog(n))$ bits.

A nossa contribuição consiste na otimização de uma estrutura de dados sucinta já existente: a Range min-Max tree (rmM-tree), proposta por \cite{paper-fully-functinal-succint-trees}. Essa estrutura fornece suporte à diversas operações sobre árvores ordinais compactas. Em sua versão estática, ela pode ser construída usando apenas $n + O(\frac{n}{b} \log n)$ bits de espaço, sendo capaz de realizar operações em tempo $O(\log n)$\citep{paper-fully-functinal-succint-trees}.
Mas mesmo ao usarmos estruturas de dados sucintas tão eficiente quanto a rmM-tree, quando trabalhamos com grandes volumes de dados, parte dos mesmos podem ser armazenados nos níveis mais inferiores da hierarquia, sobretudo quando se trabalha em um nível com capacidade de armazenamento tão pequeno quanto a cache. 
No caso da rmM-tree, temos como agravante o fato de que a mesma é uma implementação sobre árvores binárias, que por sua vez possuem fator de ramificação pequenos, o que pode gerar um alto número de transferências entre cache e RAM quando uma informação não é encontrada de imediato.

Com base no exposto, e estudos realziados, buscaremos maximizar a quantidade de informações relevantes nos nós da rmM-tree clássica, bem como o número de ramificações dessa estrutura, visando minimizar o número de eventuais trocas de dados entre cache e memória RAM.
Para tanto propomos uma versão da Range min-Max tree k-ária.

\section{Objetivos}
O objetivo central deste trabalho é propor uma versão alternativa da Range min-Max tree, visando diminuir a quantidade de transferência de dados entre níveis da memória,
melhorando assim o desempenho geral da estrutura. 

Através dessa proposta temos como objetivos secundários: estudar e compreender diferentes representações sucintas de árvores, 
seus fundamentos e benefícios; compreender os impactos da hierarquia de memória e os benefícios da utilização
adequada dos recursos fornecidos por esta arquitetura. 
Estudaremos a fundo a estrutura proposta por 
\citeauthor{paper-fully-functinal-succint-trees}, visando compreendendo melhor as operações de navegação e consulta em árvores gerais fornecidas por esta.
Implementaremos então a versão original da rmM-tree buscando abranger todas as operações suportadas na literatura, e também a versão alternativa
da rmM-tree, usando uma estrutura k-ária, afim de verificar o objetivo central do trabalho, comparando deste modo o desempenho de ambas as estruturas na hierarquia
de memória. 

Nesse sentido, faz parte também dos objetivos secundários,
compreender e aprender a utilizar corretamente bibliotecas e frameworks que ajudarão a embasar a análise experimental deste trabalho, tais como a biblioteca \textit{Succint Data Structure Library} (SDSL),
e os frameworks \textit{Google tests} e \textit{Google benchmark}.

De modo a atingir os objetivos citados, o capítulo traz um estudo sobre hierarquia de memória, estrutura de dados sucintas e as características e operações suportadas pela
range min-Max tree, o capítulo 3 apresenta a rmM-tree k-ária, expondo suas principais características e diferenças em relação à estrutura clássica. O capítulo 4, 
discorre a respeito dos testes experimentais, metodologia usada para os mesmos e resultados alcançados, por fim o capítulo 5 expões as nossas conclusões em relação 
aos objetivos e os resultados, bem como as nossas perspectivas em relação ao que foi desenvolido até o momento de escrita deste trabalho.
%contextualizar a representação sucinta de dados, falar de maneira menos técnica questões de memória. Estruturas sucintas de modo geral, uma delas são as árvores sucintas, trazer exemplos de aplicações. Para que servem as árvores, que tipo de aplicação são usadas.
%deixsr claro a proposta
%manter objetivos

%\figref{fig:research-methodology-thesis} shows
%the multimethod research design applied in this thesis.
%\begin{figure}[h]
%\centering
%  \caption[Research methodology.]{The research methodology applied for this
%  thesis.}
%  \includegraphics[width=\columnwidth]{images/research-methodology-thesis.pdf}
%  \footnotesize{Source: Made by the author.}
%  \label{fig:research-methodology-thesis}
%\end{figure}

