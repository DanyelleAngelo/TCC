The development of new real-time applications and the crescent increase data production combined with the disparity in performance between  processor and memory is a major challenge for softwares designers and developers. In this context, it becomes essential to effectively use the higher levels of hierarchy of memory, where the time spent to complete a processor  request is less and the storage capacity is reduced;  the effective use of this hierarchy can take place through the succinct data structures, which allow the representation and operation on a set of data efficiently,  while allowing the management of this data in faster memories and with  less storage capacity. The Range min-Max tree (rmM-tree) is an example of the success of these structures, built in the form of a complete binary tree, the rmM-tree occupies  about $n + O (\frac {n} {b} \log n)$ bits, and makes it possible to perform operations in time $O(\log n)$.  Although this structure has theoretically satisfactory computational cost, due to its low branching factor, a large number of data transfers between  cache and RAM can occur. This work focuses on the proposal of a min-Max tree k-ary range, aiming a higher branching factor, as a result, generating a better use of the principle of spatial locality, contributing to the improvement of the performance of the operations supported by the rmM-tree.

\begin{keywords}
Succinct data structures; Range min-Max tree; Trees; Processor-memory development gap; Cache.
\end{keywords}