O desenvolvimento de novas aplicações em tempo real e o crescente aumento na produção de dados aliados à disparidade de desempenho entre processador 
e memória é um grande desafio para projetistas e desenvolvedores de software. Neste contexto, torna-se fundamental a utilização eficaz dos níveis superiores da hierarquia de memória, onde o tempo gasto para concluir uma  solicitação do processador é menor e a capacidade de armazenamento é  reduzida; essa utilização eficaz pode acontecer por intermédio das estruturas 
de dados sucintas, pois estas possibilitam a representação e operação sobre um conjunto de dados de maneira eficiente, ao mesmo tempo em que possibilitam  o seu gerenciamento em memórias mais rápidas e com capacidade de armazenamento menor. A range min-Max tree (rmM-tree) é um exemplo de sucesso dessas estruturas, construída na forma de uma árvore binária completa, a 
rmM-tree ocupa cerca de $n + O(\frac{n}{b} \log n)$ bits, e possibilita a realização de operações sobre os objetos em  tempo $O(\log n)$. Embora essa estrutura possua custo computacional teoricamente  satisfatório, devido ao seu baixo fator de ramificação, podem ocorrer um grande número de transferências de dados entre cache e memória RAM.  Este trabalho se concentra na proposta de uma estrutura range min-Max tree k-ária, visando um maior fator de ramificação, gerando em decorrência disso um melhor aproveitamento do príncipio de localidade espacial, contribuindo para a melhoria do desempenho das operações suportadas pela rmM-tree. 


\begin{keywords}
Estrutura de dados sucintas; Range min-Max tree; Árvores; Lacuna de desenvolvimento procesador-memória; Cache.
\end{keywords}